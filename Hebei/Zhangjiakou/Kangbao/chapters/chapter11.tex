\chapter{地理位置}
\section{境域位置}
康保县位于河北省西北部,地处内蒙古高原的东南缘,属阴山穹折带,俗称“坝上高原”。
全县地理应标为东经$114^{\circ}11^{\prime}-114^{\circ}56^{\prime}$,北纬$41^{\circ}25^{\prime}-42^{\circ}08^{\prime}$之间。
东界内蒙古自治区太仆寺旗,县城距县界26.5公里;东北靠正镶白旗,县城距县界36公里,北与西北部与化德县相邻,县城距县界26 公里;西与西南部和商都县毗邻,县城距县界36 公里;南与河北省张北县接壤,县城距县界45公里;东南与沽源县相连,县城距县界39 公里。
县境东西宽62 公里,南北长80公里。县治驻康保镇,南距省会石家庄市垂立距离 420 公里,陆路交通距离 656里,东南距首都北京市垂直距离 260 公里,陆路交通距离 335 公里,南偏东路张家口市垂直距离 120 公里,陆路交通距离 135 公里。
\section{境域变化}
民国 11 年(1922 年)前,康保境域南部属张北、商都二县地,北部属察哈尔部左翼四旗及右翼正镶黄旗东半旗牧地。

民国 11 年,建置康保招垦设治局,将原张北县之马连渠克公、商都牧群等地,共计2033顷11 亩( 135.5 平方公里),划给康保设治局。
另有原察哈尔部左翼四旗及右翼正镶黄旗东半旗牧地3872.4 平方公里,划归康保设治局。东起马鞍架,西至二登图,北至胡家村,南抵孙家房,总面积4007.9 平方公里。

民国 23 年( 1934 年)5 月,将原系康保县西部及西北部第一区屯垦乡的段家地、盛兴堡、贲红察汗、五麻虎地、官围子、二登图、崔家营、车家滩、小官围子、五台坊、张先生等约28 村,612.5 平方公里土地划给化德设治局。
全县剩余面积3395.4 平方公里,折509.3 万亩,其中耕地面积302.3 万亩,林地30.1万亩,草地150.1 万亩,村庄道珞及其它占地26.8 万亩。直到1987年境域无变化。

