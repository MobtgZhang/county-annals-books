\chapter{建置沿革}
\section{地域沿革}
康保僻居朔方,沧桑屡变,历为北方各民族杂居之地。

黄帝画野分地域之前,无所稽考。

唐尧:古将中国依山川大势划分为九州(非行政区划)地域。冀州其地北直抵塞外阴山下,康保境域有阴山支脉横贯全县,故属冀州。

虞舜:分冀州西恒山之地为并州,康保仍属冀州。

夏(约公元前二十二世纪至约公元前十七世纪):将冀州划为易水至北狄。康保在其中间, \footnote{北狄即今内蒙古苏尼特旗,距北京480 公里,距张家口 280 公里,康保县距张家口 120 公里,正在其中间。}故属冀州。

商(约公元前十七世纪至约公元前十一世纪):属冀州。

周(约公元前十一世纪至公元前770 年):仍属冀州。

春秋(公元前770 年至公元前476 年):为匈奴等少数民族居住地。\footnote{匈奴冒顿单于起于阴山山脉,据有大漠南北,即从贝加尔湖到黄河南北,皆为匈奴属国。}

战国(公元前475 年至公元前221 年):属燕国地、上古郡北境。

秦(公元前221 年至公元前206 年) :康保境域属匈奴地。

西汉(公元前206年至公元 25 年):康保境域属幽州上谷郡,郡治所在沮阳(今怀来县东南)、代郡(今蔚县西北)的北境,实为匈奴帝国左王居及单于庭统治之中。

东汉(公元25 年至公元 220 年) :初为上谷郡北境,后刘秀帝弃上谷郡女祁县(今赤城县南部雕鄂一带)北境 900 里塞外,为匈奴,康保在其中。后由于丁零人联合鲜卑把匈奴驱逐出境,北匈奴西迁,南匈奴归汉,康保属南匈奴所辖之地。后乌恒兴起,据有上谷(因在大山谷之上而得名)、代那,康保为乌恒校尉所治地。鲜卑檀石槐组成鲜卑大联盟,康保又在鲜卑区域之中。

三国(公元220 年至公元265 年):北方为曹操所建之魏国,辖十一州。康保属幽州(今北京市延庆县)之北境。康保地域实为乌恒鲜卑部族活动地和游牧区。

西晋(公元265 年至公元317 年):西晋统一后,仍行州郡国并行制。设立19 州,晋末增到 21 州。康保地域仍为鲜卑活动地和游牧区。

东晋十六国时期(公元317 年至公元420 年) :当东晋偏安于江南时,北方陷于长期分裂割据的局面。康保地域先属鲜卑部建立的代国。后公元 407 年属拓跋之北魏的上谷郡。

南北朝时期(公元420 年至公元589 年) :北朝各国基本上实行州郡制。初期康保地域为高本、丁零人活动地区,后属北魏怀荒镇(驻今张北县)防地。北魏亡后,康保先属东魏幽州地,后属北齐北燕州之北境,最后又属北周,仍为北燕州之北境。

隋代(公元581 年至公元618 年) :康保地属涿州的怀戎县(治今涿鹿保岱一带)。后属雁门郡北境。

唐代(公元618 年至公元907 年) :行政区仍因袭隋制。康保地属突厥地域,求属单于都护府的桑干都督府辖(突厥失败后,移居西方,即今土尔其之祖)。

后梁(公元907 年至公元932 年) :康保为契丹活动地区。

后唐(公元923 年至公元936 年) :亦属契丹活动地区。

后晋(公元936 年至公元947 年) :仍属契丹活动地区。

后汉(公元 947 年至公元951 年) :还属契丹活动地区。

后周(公元951 年至公元960 年) :仍为契丹活动地区。

北宋(公元 960 年至公元1127 年) :康保地域属奉圣州地带,仍为契丹辽地。

辽代(公元907 年至公元1125 年) :康保境域属西京道奉圣州,治所在永兴(今涿鹿县)。

金代(公元1115 年至公元1234 年):康保前属西京路,恒州威远军节度使地(治所在今内蒙古自治区正兰旗),后直西京珞抚州镇宁军节度使地(治所在今张北县城),康保为其辖地。

元代(公元1206 年至公元1368 年):康保境域属直隶中书省隆兴路,皇庆元年( 1312 ) 10 月改为兴和路,路的治所在高原(今张北县),宝昌州所辖。

元代对于蒙古始终视为根据地,不肯放弃,而察北一带草原,亦于此时浙渐开发。

明代(公元1368 年至公元1614 年):康保境地隶属京师大宁都司,初为开平卫之兴和守御千户所(治所在今张北城),后陷于鞋在诸部落驻牧地。

奈明代二百余年间,不能完全征服内蒙元代之后裔,在明代的始终,为乱不衰,旧日移殖之民,只得弃地内迁,塞外遂又荒废。明代北方边患,自与近代建置之晚有关。

清代(公元1644 年五公元1911 年):康保境域属直隶省口北道,系东翼四旗、西翼正黄半旗游牧地。雍正二年( 1724 )7月,设直张家口理事同知厅,治张家口外民地。光绪七年改为抚民同知,属宣化府,辖康保地域,为镶黄旗牧地。

光绪二十八年( 1902 ),正式允许蒙古王公放荒招垦,自此蒙人始肯自动开放牧地,招来汉人。

正黄旗牛羊群(相似特别区)第一次迁移在乾隆年间,由殷子川移至三、四台(在张北境内)西南地方(在今康保县邓油坊一带)。第二次迁移在光绪三十二年,迁至明安牛羊群,后改为牧场(今康保县明安滩一带)。第二次迁移在民国 6 年( 1917 ),迁至正镶白旗东南地方(今康保县境内黄城子一带)。

民国元年( 1912 ),仍袭清制,为直隶省口北道统辖,康保境域属宣化府的口北三厅之张家口厅(今张北县)。

民国2年(1913)6月,废府州,成立察哈尔特别区(察哈尔,蒙语译为遥远之意),直都统辖各旗县, 7月直兴和道辖各县,康保为张北、商都二县地方。

民国3年( 1914 ),改张家口厅(清光绪年间移至于张北县城)为张北县,康保地域之南部属之。

\section{建制沿革}
民国 11 年( 1922)4月,由张北、商都两县析置康保招垦设治局,驻今康保镇。将张北县之马连渠克公、商都牧群等地和原察哈尔部左翼四旗及右翼正镶黄旗东半旗牧地划归康保设治局。

民国 14 年( 1925) 3月,改设康保招垦设治局为康保县,仍属察哈尔特别区。

民国 17 年( 1928)9月 17 日,察哈尔特别区撤销,设主察哈尔省,省会张家口,康保县属之。

民国 22 年( 1933)2 月,日本进攻热河,威胁冀察,5月间占领了康保县。5月 26 日,在共产党的推动和帮助下,冯玉祥、方振武和吉鸿昌将军(共产党员)等在张家口成立察哈尔民众抗日同盟军。6月 20 日,同盟军挥师北进, 6月22 日收复康保县, 23 日由康保移师东进。遂由国民党展保县警备司令阮玄武占领。

民国 23 年( 1934) 5月,在嘉卜寺设置化德设治局,将康保县西部及西北部划给化德设治局。

民国 24 年( 1935)12 月,察哈尔事变, 日军土肥原和秦德纯(察哈尔民政厅长)协定,长城以北宋哲元(国民党二十九军军)撤退。日本关东军协同伪满洲国蒙古军(司令李守信), 28 日侵占康保县。

民国 25 年( 1936 年)2月1日,蒙古德王(德穆楚克栋鲁普)以“蒙政会”名义在张北县成立伪察哈尔盟公署;5月 20 日,蒙奸德王将化德县(嘉卜寺)改为“额尔德木索雅勒国浩特”,并于此成立“蒙古军政府”,均辖康保县。

民国 26 年( 1937 年) 10月 28 日,日、伪在归绥(由绥远改)成立伪“蒙古联盟自治政府”下辖察哈尔盟的康保县。同年11 月21 日,日本侵略者将“蒙古联盟自治政府”、 “察南自治政府”、 “晋北自治政府”三个伪政权合并为“蒙疆联合委员会”,辖康保县。

民国 28 年( 1939 年)9月1 日,改“蒙疆联合委员会”成立伪“蒙疆联合自治政府”,首府张家口,定当年为成吉思汗纪元七百三十四年。以上伪“政府”均辖察哈尔盟的康保县。

1945 年8月8日,苏联对日宣战。 10 日蒙古人民共和国对日宣战。 12 日苏军普列耶夫上将指挥苏、蒙联军进入察北。 15 日, 日本天皇宣布无条件投降。8月22 日,康保县第一次解放,解民倒悬。察北骑兵支队在方诚、吴广义的率领下挺进康保城,共产党正式接管,建立康保县人民政府,驻今康保镇,属冀察区十九专区。

抗日战争胜利以后,恢复了察哈尔省建制, 1945 年12 月康保县改属察哈尔省察北专区(第七专区)。

1946 年9月,国民党背信弃义,挑起第二次国内战争,向共产党占领的解放区大举进攻。 11 月11 日,中共康保县委、县政府、县大队等党政机关实行战略转移,撤离县城,以内蒙古大草原为依托,游击康保南部,创建北方革命根据地。 11月 12 日,国民党傅作义部十四纵队和骑五旅一部以及曹凯、宋殿元队伍占领了康保县城,统治全县,隶属国民党察哈尔省。形成了国共两党的两套政权,共产党以农村包围城市的战咯,养精蓄锐,发反力量,迎接解放。

1947年8 月,中共察哈尔省撤销,察北专区并入冀热察行政公署,划归东北行政委员会领导,康保县属之。 9月中共察北地委将康保、商都、化德三县合并为商化康联合县,驻正镶白旗张盖湾。

1948 年12 月,著名的平津战役打响后, 26 日,国民党曹凯团、董尚清团、宋殿元团,又在察北人民解放军连连取胜下逃遁。 12月28 日,中国人民解放军骑兵十一师、十六师协同商化康联合县武工队一起入城,占领国民党军政机关和仓库要地。康保县获得第二次解放,救民于苦难之中。从此结束了国共两党的两套政权,为共产党执政。

1949年1月 15 日,冀热察区和北岳区合并,成立察哈尔省。撤销商化康联合县,恢复康保县建制,县人民政府迁回康保镇,属察哈尔察北专区。

1952年11月15 日,根据中央人民政府委员会第十九次会议决议,将察哈尔省撤销,原察南专区、察北专区合并,改为张家口专区,康保县属之,划归河北省。

1954年6月16 日,康保县人民政府改称康保县人民委员会(简称县人委)。

1958年 11年29 日,康保县并入张北县。

1959年5月,张家口专区与张家口市合并,称张家口市,辖张北县,康保区域属之。

1961年5月15 日,张家口市与张家口专区分设;从张北县以原辖区划出康保县,恢复县制,仍属张家口专区。

1967 年,张家口专区改称张家口地区,辖康保县。 12年28 日,成主康保县革命委员会,原康保县人民委员会及其所属机构撤销。 1982年9月 14 日,建立康保县人民政府,原县革命委员会撤销。

至此,几经变化更替,康保县沿革至1987年。

\section*{附录}
\begin{enumerate}
    \item 康保县建置沿革表(如表\ref{tab:part1-12-JianZhiYanGeTable}所示)
    \begin{table}[ht]
        \centering
        \begin{tabular}{p{0.3\textwidth}<{\centering}p{0.3\textwidth}<{\centering}p{0.3\textwidth}<{\centering}}
            \toprule[1.5pt]
            时期                             & 名称              & 隶属             \\
            \midrule[0.75pt]
            中华民国14年(1925年)                 & 康保县             & 察哈尔特别区         \\
            日伪侵占时期1935年12月28日至1945年8月21日   & 康保县             & 察哈尔盟公署         \\
            第一次解放时期1945年8月22日至1946年11月11日  & 康保县             & 察哈尔省察北专区       \\
            国民党占领时期1946年11月12日至1948年12月26日 & 康保县             & 察哈尔省           \\
            1947年9月至1949年1月15日             & 商(都)化(德)康(保)联合县 & 冀热察行政公署东北行政委员会 \\
            第二次解放时期1948年12月28日             & 康保县             & 察哈尔省察北专区       \\
            1958年11月29日至1961年5月            & 张北县             & 河北省张家口专区张北县    \\
            1961年5月至1987年                  & 康保县             & 河北省张家口地区      \\
            \bottomrule[1.5pt]
        \end{tabular}
        \label{tab:part1-12-JianZhiYanGeTable}
    \end{table}
    \newpage
    \item 康保县名来历考
    
    康保县以自然地理实体命名。因城南有一水淖,蒙语“康巴诺尔”,意即美丽的湖泊,康保一名取“康巴”之谐音,县名即由此而得。
\end{enumerate}