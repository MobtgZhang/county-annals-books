\chapter{行政区划}
民国 14 年( 1925 ),康保设县,将康保三千多平方公里的土地划分为4区,80 乡, 397 个闾, 1985 邻(区乡的名称尚缺无考)。

民国 23 年( 1934 ),将康保西部及西北部,原系康保县第一区屯垦乡的一部分,约28 个村, 612.5 平方公里土地划给化德设治局。

民国 24 年( 1935 ),康保县有1镇、4区、 51 乡。第一区辖1镇7乡;第二区辖17乡;第二区辖18乡;第四区辖9乡(见表)。

\begin{table}[ht]
    \centering
    \begin{tabular}{p{0.2\textwidth}<{\centering}p{0.75\textwidth}<{\centering}}
        \toprule[1.5pt]
        区名  & 所辖乡名                                                                       \\
        \midrule[0.75pt]
        第一区 & 城关镇、卧虎乡、道德乡、百忍乡、五福乡、丹田乡、白龙乡、屯垦乡。                                           \\
        第二区 & 二聚乡、戈家乡、白脑包乡、傅大乡、五善人乡、立德乡、郝家乡、王家乡、富常乡、嘉德乡、富登乡、聚贤乡、三义堂乡、十大股乡、西富贵乡、郑家乡、忠义乡。  \\
        第三区 & 丹清乡、青海乡、水泉乡、新庙乡、义合乡、南通乡、宫升乡、太平乡、公义乡、北达乡、福隆乡、京宝乡、东 顺乡、万隆乡、合义乡、哈明乡、东宫贵乡、平香乡。 \\
        第四区 & 宝丰乡、土城乡、平安乡、小庙乡、天城乡、麻呢乡、聚富乡、三顺乡、丰泰乡。                                       \\
        \bottomrule[1.5pt]
    \end{tabular}
\end{table}

1936 年日本帝国主义侵占康保县,康保县划为1镇、5区、 17 乡。

1947 年国民党占领时期,康保县设1镇、17 个乡(乡名参看基层政权)、127 个保、 1460 个甲。

1949 年1月第二次解放后,全县共辖1市6区、243 个行政村、 628 个自然村。1市为康保市。6区为:屯垦区、郝车倌区、土城子区、 邓油坊区、丹木淖区、满德堂区。

1949 年4月,全县分成7个工作区。7个工作区为:城关区、郝车倌区、土城子区、邓油坊区、丹木淖区、满德堂区、屯垦区。

